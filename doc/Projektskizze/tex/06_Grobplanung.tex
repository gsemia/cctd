\section{Grobplanung}
Die Gesammtlaufzeit des Projekts CCTD wird durch externe Vorlagen festgelegt. Das Projekt startete am 24.09.2010 und endet gemäss Abgabetermin am 10.12.2010. Dies ergibt eine Gesammtlaufzeit von 12 Wochen. Geplant ist nur ein Release mit der gesamten geplanten Funktionalität des Spiels. Die Planung erfolgt gemäss UP mit einer Iterationsdauer von 2 Wochen:


\begin{tabular}{ | l | l | l | p{9.5cm} |}
\hline
Phase & Woche & Iteration & Ziele \\ \hline \hline
Inception & 38 &  1  & Projektidee gefunden, Projekt Ausrichtung definiert, Gruppenbildung, Projektleader definiert \\ \hline
& 39 &    & Projektskizze erstellt, Entwicklungs Umgebung definiert, Projektskizze Präsentation erstellt \\ \hline
Elaboration & 40 &  2  & \textcolor{red}{Abgabe Projektskizze}, \textcolor{red}{Präsentation Projektskizze}, Spielprinzip definiert, Applikations Grundgerüst erstellt \\ \hline
 & 41 &    & Anwendungsfälle detailiert ausformuliert, Rest identifiziert und priorisiert, 1. Entwurf Domänenmodell \\ \hline
 & 42 &  3  & Use Cases fertig, Supplementary Specification fertig falls benötigt, System-Sequenzdiagram erstellt, Glossar erstellt, Vision fertig, Domänenmodell erstellt, Erste Java Tests bezüglich Client / Server verhalten; welche Daten werden übertragen \\ \hline
 & 43 &    & \textcolor{red}{Abgabe Analyse}, \textcolor{red}{Präsentation Analyse}, erste Version UML Klassendiagram,  erste Version Architekturdokumente   \\ \hline
Construction & 44 &  4  & UML Klassendiagram, UML Interaktionsdiagram für ausgewählte Systemoperationen, Architekturdokumente fertig, Implementation des UML Klassendiagrams in Java, Test der Zusammenhänge: ev. Klassendiagram überarbeiten \\ \hline
 & 45 &    & UML Klassendiagram, UML Interaktionsdiagram für ausgewählte Systemoperationen, Architekturdokumente fertig,  Design Präsentation fertig, erste Implementation der Klassen inklusive Methoden und Eigenschaften \\ \hline
 & 46 &  5  & \textcolor{red}{Abgabe Design}, \textcolor{red}{Präsentation Design}, Implementation der einzelnen Klassen, Client / Server Testing, Implemtation einer simplen Spieloberfläche mit ersten Elementen  (ein Turm, ein Gegner, vier Spieler). \\ \hline
 & 47 &    &  Simple Spieloberfläche fertig, Testing mit Client / Server, Ein Spiel von A-Z Durchspielen sollte ab hier möglich sein, Spiel erweitern auf unterschiedliche Türme, Unterschiedliche Gegnertypen, Unterschiedliche Gegnerwellen, Testing der ganzen Klassen \\ \hline
Transition & 48 &  6  & System Test fertig, Schlussdokumentation erstellen, Schlusspräsentation erstellt, Spiel fertig stellen, Alle Spielelemente fertig, Spiel Ballancing (Schwierigkeit: nicht zu einfach, nicht zu schwer) \\ \hline
 & 49 &   & Projekt Ende, \textcolor{red}{Schlusspräsentation}, \textcolor{red}{Alles abgeben} \\
 \hline
 \end{tabular}


