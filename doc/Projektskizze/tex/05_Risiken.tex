\section{Risiken} \label{sec:Risiken}
\subsection{Verwertungsrisiken}
\subsubsection{Nachfrage} \label{subsec:Risiken-Nachfrage}
Nach einer kurzen Suche mit Google ergeben sich etliche Treffer für Tower Defense Spiele in etlichen Formen. Zum Beispiel towerdefence.net listet 196 unterschiedliche Tower Defense Implementationen (Stand: 01.10.2010). Ausserdem unterstützen mehrere Multiplayer Strategy Spiele einen Tower Defense Modus (Warcraft 3, Starcraft 2, ...). \\
Ein weiteres Problem stellt die Gewinngenerierung dar. Die meisten Tower Defense Spiele die sind entweder eine Erweiterung zu einem bereits existierenden Spiel oder können Gratis heruntergeladen werden.

\subsection{Projektrisiken}
\subsubsection{Kosten} \label{subsec:Risiken-Kosten}
Für dieses Projekt werden pro Person 120 Arbeitsstunden geplant. Unsere Gruppe besteht aus 5 Personen, dies ergeben 600 Arbeitsstunden. Bei einem Stundenansatz von 60 Franken pro Stunde ergibt dies erwartete Entwicklungskosten von 36'000 Franken. Um diese Kosten innerhalb von 3 Jahren zu decken, müsste durchschnittlich 12'000 Franken pro Jahr eingenommen werden. \\
Wie bereits im Abschnit \nameref{subsec:Risiken-Nachfrage} beschrieben, wird es in diesem Markt schwierig, ein neues Produkt zu präsentieren und die Entwicklungskosten zu kompensieren.

\subsubsection{Projektumfang} \label{subsec:Risiken-Projektumfang}
Der Projektumfang ist für die kurze Zeit von einem Semester gross gehalten. Deswegen kann es schwierig werden das ganze Produkt mit allen vorgesehenen Features in der kurzen Zeit zu implementieren. \\
Stellt sich im Verlaufe des Projektes heraus, dass das Projekt nicht im gegebenen Zeitrahmen zum vollständigen Abschluss gebracht werden kann, werden gewisse Features weggelassen.

\subsection{Allgemeine Risiken}
\subsubsection{Personalausfall} \label{subsec:Risiken-Personalausfall}
Es besteht das Risiko, dass ein oder mehrere Projektmitglieder schwer erkrankt oder durch Militäreinsätze für einen beträchtlichen Zeitraum während dem Projekt ausfallen. \\
In dieser Situation müsste der Projektumfang gekürzt werden, damit im gleichen Zeitraum das Projekt zu einem Abschluss gebracht werden kann.

\subsubsection{Hardwareausfall} \label{subsec:Risiken-Hardwareausfall}
Während der ganzen Projektdauer werden mehrere Computer eingesetzt. Während der Entwicklung der Software besteht eine Möglichkeit, dass eine Harddisk ausfällt und mit dieser Daten, welche noch nicht ins SVN-Repository hochgeladen wurden. \\
Während dem Projekt müssen regelmässig Backups von dem SVN-Repository erstellt werden und jeder ist dafür verantwortlich, dass die persönlichen Änderungen regelmässig und in kurzen Abständen ins SVN-Repository hochgeladen werden.

