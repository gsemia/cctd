
\documentclass[a4paper, headsepline, footsepline, oneside]{scrartcl}

\usepackage{graphicx}
\usepackage{color}
\usepackage{xcolor}
\usepackage{listings}
\usepackage[latin1]{inputenc}
\usepackage[ngerman]{babel}
\usepackage{pdfpages}
\usepackage[plainfootsepline ]{scrpage2}
\usepackage{url}
\definecolor{lightgray}{gray}{0.5}
\setlength{\parindent}{0pt}
% Seitenh�he anpassen
\usepackage[height=680pt, width=500pt]{geometry}
\usepackage {booktabs}
\usepackage[colorlinks,pdfpagelabels,pdfstartview = FitH,bookmarksopen = true,bookmarksnumbered = true,linkcolor = black,plainpages = false,hypertexnames = false,citecolor = black] {hyperref}
\hypersetup{pdftitle={Protokoll - Feedback Projektskizze-Dokument und -Pr�sentation}}
% Kopf und Fusszeilen

\renewcommand*{\titlepagestyle}{scrheadings}
\pagestyle{scrheadings}
\headheight 50 pt

% Header
%\ihead{\sf{\hspace{7mm} \huge ...}}
\ohead{\includegraphics[scale=0.4]{zhaw_logo_de.jpg}}

% Footer
\ifoot{ZHAW / SEP: Feedback Projektskizze}
\ofoot{R.Koch, L.Spirig, B.Felder, F.Eriksson, N.Koch}

\begin{document}

\author{Rolf Koch, Lukas Spirig, Benjamin Felder, Fabian Eriksson, Nathanael Koch}
\subject{SWE / SEP}
\title{Protokoll: Feedback Projektskizze}
\maketitle

\begin{minipage}{0.5\textwidth}
\begin{flushleft}
    \section{Protokoll 15.10.2010}
\end{flushleft}
\end{minipage}
\begin{minipage}{0.5\textwidth}
\begin{flushright}
    \textbf{Note: 5.5} \small{inkl. Pr�sentation}\\[1.5cm]
\end{flushright}
\end{minipage}
\\
\subsection{Idee} Ok, Die Abgrenzung zu anderen vorhandenen Tower Defense Spielen ist nicht klar genug beschrieben.
\subsection{Hauptanwendungsfall} gut, detailiert, klar / "�berstehen die Spieler alle Runden ist das Spiel gewonnen.", sollte genauer definiert werden.
\subsection{Weitere Anforderungen} gut / Was passiert, wenn 1,2 oder 3 Spieler spielen? (Unklar). / Punkt 3 k�nnte besser geschrieben werden.
\subsection{Risikoanalyse} gut / Kosten sind ein Verwertungsrisiko und kein Projektrisiko / Personen- und Hardware-Risiko sind normale Risiken und sind nicht n�tig aufzulisten.
\subsection{Grobplannung} gut, detailiert und doch allgemein gehalten
\subsection{Kundennutzen} soziale Aspekte sollten zwingender festgelegt werden / Abgrenzung muss besser festgelegt werden.
\subsection{Wirtschaftlichkeit} ok 
\subsection{Misc} Was ein \textit{creep} ist muss unbedingt im Glossar definiert werden.

\end{document}

