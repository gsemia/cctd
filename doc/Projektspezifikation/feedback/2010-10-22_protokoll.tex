
\documentclass[a4paper, headsepline, footsepline, oneside]{scrartcl}

\usepackage{graphicx}
\usepackage{color}
\usepackage{xcolor}
\usepackage{listings}
\usepackage[latin1]{inputenc}
\usepackage[ngerman]{babel}
\usepackage{pdfpages}
\usepackage[plainfootsepline ]{scrpage2}
\usepackage{url}
\definecolor{lightgray}{gray}{0.5}
\setlength{\parindent}{0pt}
% Seitenh�he anpassen
\usepackage[height=680pt, width=500pt]{geometry}
\usepackage {booktabs}
\usepackage[colorlinks,pdfpagelabels,pdfstartview = FitH,bookmarksopen = true,bookmarksnumbered = true,linkcolor = black,plainpages = false,hypertexnames = false,citecolor = black] {hyperref}
\hypersetup{pdftitle={Protokoll - Feedback Projektspezifikation}}
% Kopf und Fusszeilen

\renewcommand*{\titlepagestyle}{scrheadings}
\pagestyle{scrheadings}
\headheight 50 pt

% Header
%\ihead{\sf{\hspace{7mm} \huge ...}}
\ohead{\includegraphics[scale=0.4]{zhaw_logo_de.jpg}}

% Footer
\ifoot{ZHAW / SEP: Feedback Projektspezifikation}
\ofoot{R.Koch, L.Spirig, B.Felder, F.Eriksson, N.Koch}

\begin{document}

\author{Rolf Koch, Lukas Spirig, Benjamin Felder, Fabian Eriksson, Nathanael Koch}
\subject{SWE / SEP}
\title{Protokoll: Feedback Projektspezifikation}
\maketitle

\section{Protokoll 22.10.2010}
\subsection{Anwendungsall-Diagramm / UseCase-Diagramm}
UseCases sind Aktionen und sollten deswegen immer ein Verb im Namen haben \\
UseCase ``Chat'' in ``chatten'' umbenennen \\
UseCase ``Spiel starten'': \begin{itemize}
\item sollte es um den Start des Spiels gehen und nicht um das ganze Spiel (Das Beenden ist nicht teil davon) 
\item evtl. aufsplitten in mehrere UseCases
\end{itemize}
\subsection{Sequenzdiagramm}
Umbenennen in System-Sequenzdiagramm \\
Komplexit�t des ``Spiel starten'' ist an der oberen Grenze bez�glich Komplexit�t \\
closeGame sollte entfernt werden
\subsection{Systemoperationen}
Operationen/Methoden: Datentypen fehlen (bei Parameter und R�ckgabewert)
\subsection{Domainmodel}
Das Domainmodel beschreibt das Problem und sollte deshalb \textbf{keine} L�sungsans�tze beinhalten. GUI kann z.B. entfernt werden. \\
Beschreibung der einzelnen Domains fehlen \\
Map $\rightarrow$ ConstructionZone: Pfeil fehlt
\subsection{Projektmanagment}
Soll- und Ist-Aufwand in Stunden sollte hinzugef�gt werden.
\subsection{Zus�tzliche Spezifikationen}
vorsichtig mit Anforderungen in diesem Dokument, diese sind schlussendlich \textbf{bindend} \\
\textbf{Spielerfocus/Kameraf�hrung}: ist f�r jetzt OK. Sp�ter evtl. in ein UseCase umwandenln \\
\textbf{Spiele integrit�t}: kann gestrichen werden \\
\textbf{Real-time Anforderungen}: m�ssen hinzugef�gt werden
\subsection{Architektur}
\begin{itemize}
\item Client / Server Konzept sollte mehr hervorgehoben werden
\item THIN- oder FAT-Client festlegen
\item sync?
\item Zusammenhang mit real-time Anforderungen
\end{itemize}
\subsection{Pr�sentation von n�chster Woche}
ca. 10 minuten, eventuell bereits mit Prototype

\end{document}

