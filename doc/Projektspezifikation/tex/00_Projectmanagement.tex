\section{Projekt Management} \label{sec:Projekt-Management}
\subsection{Aktueller Status} \label{sec:Aktueller-Status}
In dieser Sektion befindet sich ein Vergleich des geplanten Aufwands und des tatsächlich erreichten Resultats sowie dessen jeweiliger Aufwand der für die einzelnen Tasks geleistet wurde.

\begin{table}[htp]
\begin{tabular}{ | p{6.5cm} | l  | l | l | l |}
\hline
\textbf{Aufwände} & \textbf{Aufw. Geplant}  & \textbf{Aufw.} & \textbf{Aufw. p.P} & \textbf{Anz. P} \\ \hline \hline
Spielprinzip definiert & ? 0  & 15 & 3 & 5 \\ \hline
Entwicklungs Umgebung definiert & ?  & 6 & 3 & 2 \\ \hline
\textbf{Projekt Skizze:}
\begin{itemize}
\item Projekt Idee
\item Hauptanwendungsfall
\item Weitere Anforderungen
\item Ressourcen
\item Risiken
\item Grobplanung
\item Kundennutzen
\item Wirtschaftlichkeit 
\end{itemize}& ?   & 40 & 8 & 5 \\ \hline
\textbf{Präsentation Projekt Skizze} & ?  & 5 & 5 & 1 \\ \hline
Applikations Grundgerüst erstellt & 5  & \textcolor{red}{nicht erfüllt} & ? & ? \\ \hline
Anwendungsfälle detailiert ausformuliert & 8  & \textcolor{red}{12} & 6 & 2 \\ \hline
Domänenmodel Entwurf & 10 & 10 & 2 & 5 \\ \hline
Projekt Management auf neusten Stand gebracht  & 7 & 7 & 7 & 1 \\ \hline
Architektur & 5 & 5 & 5 & 1 \\ \hline
Supplementary Specification erstellt & 6  & 6 & 6 & 1 \\ \hline
System Sequenzdiagram & 8  & 8 & 8 & 1 \\ \hline
Glossar erstellt & 3  & \textcolor{red}{6} & 2 & 3 \\ \hline
Vision fertig gestellt & 4 & 3 & 3 & 1 \\ \hline
Domänenmodell erstellt & 5  & 5 & 5 & 1 \\ \hline
Java Tests bezüglich Client / Server verhalten & 12 & \textcolor{red}{nicht erfüllt} & ? & ? \\ \hline
Welche Daten werden übertragen & 3 & \textcolor{red}{nicht erfüllt} & ? & ? \\ \hline
\textbf{Analyse Dokumentation fertigstellen} & 5 & 5 & 5 & 1 \\ \hline
\textbf{Analyse Präsentation erstellt} & 3 & 3 & 3 & 1 \\ \hline
UML Klassendiagram Entwurf & 3 & 3 & 1.5 & 2 \\ \hline
\end{tabular}
\caption{Projekt Management: Aktueller Status}
\end{table}

\subsection{Massnahmen} \label{sec:Massnahmen}
Für die Design Phase wird nun vermehrt auch erster Javacode generiert. In der Iteration 3 (Analyse + Design) 
wird nun auch das Applikations Grundgerüst erstellt und erste Java Tests bezüglich Client / Server verhalten durchgeführt. Die Tasks wurden verschoben.

\subsection{Weitere Planung} \label{sec:Weitere-Planung}

\textbf{Analyse + Design (Iteration 3)}

Wir befinden uns momentan in dieser Iteration 3 und somit in der Elaborations Phase. Zu diesem Zeitpunkt geht die Analyse Phase in die Design Phase über.

\begin{table}[htp]
\begin{tabular}{ | l | l | l |  p{6.5cm} | l |}
\hline
Phase & Woche & Iteration & Task & Geplanter Aufwand \\ \hline \hline
Elaboration & 42 &  3  & Use Cases fertig & 4 \\ \hline
 &  &  & Supplementary Specification fertig falls benötigt & 3 \\ \hline
 &  &  & System-Sequenzdiagram erstellt & 3 \\ \hline
 &  &  & Glossar erstellt & 3 \\ \hline
 &  &  & Vision fertig & 4 \\ \hline
 &  &  & Domänenmodell erstellt & 3 \\ \hline
 &  &  & Erste Java Tests bezüglich Client / Server verhalten & 12 \\ \hline
 &  &  & welche Daten werden übertragen & 6 \\ \hline
 &  &  & Applikations Grundgerüst erstellt & 5 \\ \hline
 \hline
 & 43 &  & \textcolor{red}{Abgabe Analyse} & 5 \\ \hline
 &  &  & \textcolor{red}{Präsentation Analyse} & 1 \\ \hline
 &  &  & erste Version UML Klassendiagram & 15  \\ \hline
 &  &  & erste Version Architekturdokumente & 12  \\ \hline
\end{tabular}
\caption{Projekt Management: Iteration 3}
\end{table}
Woche 42: 40 Stunden. Pro Person: 8 Stunden

Woche 43: 33 Stunden. Pro Person: 6,6 Stunden

Total: 73 Stunden. Pro Person: 14,6 Stunden

\textbf{Design (Iteration 4)}

Ab Woche 44 startet die Design Phase mit der Iteration 4 bei der es hauptsächlich um das Code Design geht.

\begin{table}[htp]
\begin{tabular}{ | l | l | l |  p{6.5cm} | l |}
\hline
Phase & Woche & Iteration & Task & Geplanter Aufwand \\ \hline
\hline
Construction & 44 &  4  & UML Klassendiagram & 10 \\ \hline
 &  &  & UML Interaktionsdiagram für ausgewählte Systemoperationen & 10 \\ \hline
 &  &  & Architekturdokumente fertig 5 \\ \hline
 &  &  & Implementation des UML Klassendiagrams in Java & 15 \\ \hline
 &  &  & Test der Zusammenhänge: ev. Klassendiagram überarbeiten & 5 \\ \hline \hline
 & 45 &  & UML Klassendiagram & 15 \\ \hline
 &  &  & UML Interaktionsdiagram für ausgewählte Systemoperationen & 10 \\ \hline
 &  &  & Architekturdokumente fertig & 5 \\ \hline
 &  &  & Design Präsentation fertig & 5 \\ \hline
 &  &  & Erste Implementation der Klassen inklusive Methoden und Eigenschaften &  15 \\ \hline
\end{tabular}
\caption{Projekt Management: Iteration 4}
\end{table}

Woche 44: 45 Stunden. Pro Person: 9 Stunden

Woche 45: 50 Stunden. Pro Person: 10 Stunden

Total: 95 Stunden. Pro Person: 19 Stunden

\subsection{Anpassung der Projekt Aufwand Schätzung} \label{sec:Anpassung-der-Projekt-Aufwand-Schaetzung}

Aufgrund der nun vorliegenden Zahlen muss die eher etwas übertriebene Aufwandschätzung korrigiert werden. Geschätzt war ursprünglich ein Aufwand von 600 Mannstunden verteilt auf 5 Personen. Aktuelle Zahlen zeigen, dass dieser Aufwand voraussichtlich nicht benötigt wird.

Wir setzen daher den geschätzten Aufwand neu auf ca 450 Stunden fest. Dies entspricht etwa einem Aufwand von 90 Stunden pro Iteration Total $\rightarrow$ 45 Stunden pro Woche Total $\rightarrow$ 9 Stunden pro Person pro Woche.

\subsection{Risiken} \label{sec:Risiken}
\subsubsection{Nachfrage} \label{subsec:Risiken-Nachfrage}
Nach einer kurzen Suche mit Google ergeben sich etliche Treffer für ``Tower Defense``-Spiele in etlichen Formen. Zum Beispiel towerdefence.net listet 196 unterschiedliche ``Tower Defense``-Implementationen (Stand: 01.10.2010). Ausserdem unterstützen mehrere Multiplayer Strategy Spiele einen Tower Defense Modus (Warcraft 3, Starcraft 2, ...). \\
Ein weiteres Problem stellt die Gewinngenerierung dar. Die meisten ``Tower Defense``-Spiele sind entweder eine Erweiterung zu einem bereits existierenden Spiel oder können Gratis heruntergeladen werden.

\subsubsection{Projektumfang} \label{subsec:Risiken-Projektumfang}
Der Projektumfang ist für die kurze Zeit von einem Semester gross gehalten. Deswegen kann es schwierig werden das ganze Produkt mit allen vorgesehenen Features in der kurzen Zeit zu implementieren. \\
Stellt sich im Verlaufe des Projektes heraus, dass das Projekt nicht im gegebenen Zeitrahmen zum vollständigen Abschluss gebracht werden kann, werden gewisse Features weggelassen.

\subsubsection{Mitarbeiter Motivation} \label{subsec:Mitarbeiter-Motivation}
Da das Projekt `nur' auf Basis eines Wahlmoduls entwickelt wird und keine reale Bezahlung stattfindet, kann es passieren, dass sich einzelne Mitarbeiter des Teams nicht motivieren können für das Projekt zu arbeiten. \\
Dies kann eventuell dazu führen, dass nicht alle geplanten Tasks durchgeführt werden und Tasks immer weiter nach hinten geschoben werden, was schlussendlich zum Risiko `Projektumfang' führt und eine Reduktion der Features getätigt werden muss um das Projekt doch noch vollständig abschliessen zu können.

